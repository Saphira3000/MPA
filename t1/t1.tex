\documentclass[12pt]{article}
\usepackage[utf8]{inputenc}
\usepackage[left=2cm,right=2cm,top=2cm,bottom=2cm]{geometry}
\usepackage{graphicx, color, multicol, multirow}
\usepackage{float}
\usepackage[small]{caption}
\usepackage[spanish]{babel}
\usepackage{enumerate, amsmath, amssymb, listings}

%%%%%%%%%%%%%%%%%%%%%%%%%%%%%%%%%%%%%%%%%%%

\setlength{\parskip}{\baselineskip}
\spanishdecimal{.}

\begin{document}

%%%%%%%%%%%%%%%%%%%%%%%%%%%%%%%%%%%%%%%%%%
\thispagestyle{empty}
\hrule height2.5pt
\vspace{.1cm}
\hrule height1pt
\vspace{.3cm}

\begin{center}
\Large {An\'alisis factorial del porcentaje de ocupaci\'on en la poblaci\'on mexicana.} \bigskip \\
{ Gabriela S\'anchez Y.\\
1935064}
\end{center}

\vspace{.3cm}
\hrule height1pt
\vspace{.1cm}
\hrule height2.5pt
\vspace*{.5cm}
%%%%%%%%%%%%%%%%%%%%%%%%%%%%%%%%%%%%%%%%%%%%


\section{Introducci\'on}
Es de inter\'es conocer el estado ocupacional de una poblaci\'on ya que est\'a claro que una sociedad econ\'omicamente activa propiciar\'a la prosperidad y el desarrollo de dicha poblaci\'on.  \bigskip \\
La finalidad de este trabajo es facilitar al lector el an\'alisis de algunos de los indicadores presentes en la Encuesta Nacional de Ocupaci\'on y Empleo realizada por el Instituto Nacional de Estad\'istica y Geograf\'ia de M\'exico en el a\~no 2017. \bigskip \\
Para ello, se decidi\'o estudiar la existencia o ausencia de una relaci\'on de dependencia entre los diferentes niveles de estudios y el estado de ocupaci\'on en individuos de la poblaci\'on mexicana pertenecientes a localidades rurales y urbanos. \bigskip \\
Es decir, que se busca conocer si tanto el nivel de estudios as\'i como la zona o localidad en la que residen tienen alg\'un tipo de influencia positiva o negativa en el porcentaje de actividad ocupacional de la poblaci\'on a la que pertenecen.




\subsection{An\'alisis estad\'istico de datos}

 

\nocite{*}
\bibliographystyle{plain}
\bibliography{biblio_disenio}



\end{document}

