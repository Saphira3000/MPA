\documentclass[12pt]{article}
\usepackage[utf8]{inputenc}
\usepackage[left=2cm,right=2cm,top=2cm,bottom=2cm]{geometry}
\usepackage{graphicx, amsmath}
\usepackage[small]{caption}
\usepackage{subcaption, enumerate}
\usepackage{url}
\setlength{\parskip}{\baselineskip}
%\setlength{\parskip}{0.3cm}
\graphicspath{ {images/} }
\allowdisplaybreaks

\usepackage{hyperref}
\hypersetup{
	colorlinks=true,
	linkcolor=blue,     
	urlcolor=blue,
	citecolor=blue,
}


\begin{document}

	\thispagestyle{empty}

	\begin{center}
		{\Large \bf Generating functions}\\
		Gabriela S\'anchez Y.\\
		5064
	\end{center}
  
	In this activity some exercises of the book Introduction to Probability \cite{prob2003} are solved.
	
	{\bf Exercise 1, page 392}
	
	{\em Let $Z_1, Z_2, \ldots, Z_n$ describe a branching process in which each parent has $j$ offspring with pro\-ba\-bi\-li\-ty $p_j$. Find the probability $d$ that the process eventually dies out if 
		\begin{enumerate}[a)]
			\item $p_0 = 1 / 2, \, p_1 = 1 / 4, \, p_2 = 1 / 4$.
			\item $p_0 = 1 / 3, \, p_1 = 1 / 3, \, p_2 = 1 / 3$.
			\item $p_0 = 1 / 3, \, p_1 = 0, \, p_2 = 2 / 3$.
			\item $p_j = 1 / 2^{j + 1}$, for $j = 0, 1, 2, \ldots$
			\item $p_j = (1 / 3) (2/3)^j$, for $j = 0, 1, 2, \ldots$
			\item $p_j = e^{-2} 2^j / j!$, for $j = 0, 1, 2, \ldots$ (estimate $d$ nummerically).
		\end{enumerate}
	}

	%To solve this exercise, it is important recall some concepts first. Let $d$ the probability that the process will ultimately die out. By 
	Let $d$ the probability that the process will ultimately die out. Theorem 10.2 from page 380 \cite{prob2003} says that if the mean number $m$ of offspring produced by a single parent is $\leq 1$, then $d = 1$ and the process dies out with probability 1. But if $m > 1$ then $d < 1$ and the process dies out with probability $d$.
	
	In the particular case of a), b) and c), the mean number $m$ of offspring produced by a single parent is $m=p_1 + 2p_2 = 1-p_0+p_2$. If $m > 1$, $d$ can be easily calculated by $d= p_0/p_2$. 
	
	\noindent a) $p_0 = 1 / 2, \, p_1 = 1 / 4, \, p_2 = 1 / 4$
	
	The mean number $m$ of offspring produced by a single parent is 
	$$m = \frac{1}{4} + 2\left( \frac{1}{4} \right)= \frac{3}{4} < 1.$$
	Then, by theorem 10.2, follows that the process dies out with probability 1.
	
	\noindent b) $p_0 = 1 / 3, \, p_1 = 1 / 3, \, p_2 = 1 / 3$
	
	For this exercise $m = \frac{1}{3} + 2\left(\frac{1}{3} \right) = 1$. Therefore the process dies out with probability 1.
	
	\noindent c) $p_0 = 1 / 3, \, p_1 = 0, \, p_2 = 2 / 3$
	
	The mean number $m$ of offspring produced by a single parent in this case is
	\begin{equation*}
	m = 0 + 2\left( \frac{2}{3} \right) = \frac{4}{3} > 1.
	\end{equation*}
	The process dies out with probability $d= p_0/p_2 = \frac{1}{3}/\frac{2}{3} = 0.5$.
	
	To solve d) and e) it is neccesary to remember that $h(z)$, the ordinary generating function for the $p_i$, is 
	$$h(z) = p_0 + p_1 z + p_2 z^2 + \cdots $$
	and $m=h'(1)$. If $m\leq 1$, the process will surely die out and $d = 1$. To find the
	probability $d$ when $m > 1$ one must find a root $d < 1$ of the equation
	\begin{equation*}
	z = h(z).
	\end{equation*}
	
	\noindent d) $p_j = 1 / 2^{j + 1}$, for $j = 0, 1, 2, \ldots$
	
	The ordinary generating function of the problem is 
	\begin{eqnarray*}
	h(z) &=& \frac{1}{2} + \frac{1}{2^2}z + \frac{1}{2^3}z^2 + \ldots \\
	&=& \frac{1}{2} \left( 1 + \frac{1}{2} z + \frac{1}{2^2}z^2 + \ldots \right) \\
	&=& \frac{1}{2} \left[ \left(\frac{1}{2}z\right)^0 + \left(\frac{1}{2}z\right)^1 + \left(\frac{1}{2}z\right)^2 + \ldots \right] \\
	&=& \frac{1}{2} \left(\frac{1}{1 - \frac{1}{2} z}\right)\\
	&=& \frac{1}{2 - z}.
	\end{eqnarray*}

	Recall that a geometric serie 
	
	{\bf Excercise 3, page 392}
	
	{\em In the chain letter problem (see Example 10.14) find your expected profit if {\setlength{\parskip}{0cm}
			\begin{enumerate}[a)]
				\item $p_0 = 1/2, \, p_1 = 0, \, p_2 = 1/2$.
				\item $p_0 = 1/6, \, p_1 = 1/2, \, p_2 = 1/3$.
			\end{enumerate}
			Show that if $p_0 > 1 / 2$, you cannot expect to make a profit.}
	}
	
	The expected profit of the chain letter problem can be found by the expression $50m + 50 m^{12},$ where $m = p_1 + 2p_2$.
	
	\noindent a) $p_0 = 1/2, \, p_1 = 0, \, p_2 = 1/2$.

	In this particular case $m = 0 + 2\left(\frac{1}{2}\right) = 1$. Then, the expected profit is: $50(1+1^{12})-100 =0$.

	\noindent b) $p_0 = 1/6, \, p_1 = 1/2, \, p_2 = 1/3$.
	
	For this problem $m = \frac{1}{2}+ 2\left(\frac{1}{3}\right) = \frac{7}{6}$ and the expected profit is $$50 \left[ \frac{7}{6}+\left(\frac{7}{6} \right)^{12} \right]-100 \approx 376.26 - 100 = 276.26.$$
	
	

	{\bf Exercise 1, page 401}
	
	{\em Let $X$ be a continuous random variable with values in $[0,2]$ and density $f_X$. Find the moment generating function $g(t)$ for $X$ if
		\begin{enumerate}[a)]
			\item $f_X(x) = \frac{1}{2}$.
			\item $f_X (x) = \frac{1}{2}x$.
			\item $f_X (x) = 1 - \frac{1}{2}x$.
			\item $f_X (x) = |1 - x|$.
			\item $f_X (x) = \frac{3}{8}x^2$.
		\end{enumerate}
	}
	
	The {\em moment generating function $g(t)$} for $X$ is define by Equation (\ref{mgf})
	\begin{equation}
	g(t) = \int_{-\infty}^{\infty} e^{tx} f_X(x) \, dx.
	\label{mgf}
	\end{equation}
	
	The values of the variable $X$ are in the interval $[0,2]$, therefore the moment generating funcion will be define by the integral in Equation (\ref{mgf_exercise}) 
	\begin{equation}
		g(t) = \int_{0}^{2} e^{tx} f_X(x) \, dx.
	\label{mgf_exercise}
	\end{equation}
	
	\noindent a) $f_X(x) = \frac{1}{2}$.
	\begin{eqnarray*}
		g(t) &=& \int_{0}^{2} e^{tx} \left(\frac{1}{2}\right)\, dx \\
		&=& \frac{1}{2} \int_{0}^{2} e^{tx} \, dx \\
		&=& \frac{1}{2} \left. \left[ \frac{1}{t} \cdot e^{tx} \right] \right|_{0}^{2} \\
		&=& \frac{1}{2}	\cdot \frac{e^{2t}-1}{t}.
	\end{eqnarray*}

    \noindent b) $f_X(x) = \frac{1}{2}x$.
	\begin{eqnarray*}
		g(t) &=& \int_{0}^{2} e^{tx} \left(\frac{1}{2} x \right)\, dx \\
		&=& \frac{1}{2} \int_{0}^{2} xe^{tx} \, dx \hspace{3cm} \text{i.b.p} \\
		&=& \frac{1}{2} \left. \left[ \frac{x}{t} \cdot e^{tx} - \frac{1}{t^2} \cdot e^{tx}\right] \right|_{0}^{2} \\
		&=& \frac{1}{2} \cdot \frac{2te^{2t} - e^{2t} + 1}{t^2}.
	\end{eqnarray*}

	\noindent c) $f_X (x) = 1 - \frac{1}{2}x$.
	\begin{eqnarray*}
		g(t) &=& \int_{0}^{2} e^{tx} \left( 1 - \frac{1}{2}x \right)\, dx \\
		&=& \int_{0}^{2} e^{tx} \, dx - \int_{0}^{2} e^{tx} \left(\frac{1}{2} x \right)\, dx \\
		&=& \int_{0}^{2} e^{tx} \, dx - \frac{1}{2} \int_{0}^{2} xe^{tx} \, dx 
	\end{eqnarray*}

	Note that these integrals were already calculated in a) and b), then
	\begin{eqnarray*}
		g(t) &=& \int_{0}^{2} e^{tx} \, dx - \frac{1}{2} \int_{0}^{2} xe^{tx} \, dx \\
		&=& \frac{e^{2t}-1}{t} - \frac{1}{2} \cdot \frac{2te^{2t} - e^{2t} + 1}{t^2} \\
		&=& \frac{3te^{2t} - 3t - 2te^{2t} + e^{2t} - 1}{2t^2} \\
		&=& \frac{e^{2t} - 2t + 1}{2t^2}.
	\end{eqnarray*}

	\noindent d) $f_X (x) = |1 - x|$.

	Following the definition of absolute value, the density function $f_X$ can be define by Equation (\ref{absval})
	\begin{equation}
	f_X(x)= \left\{ \begin{array}{lcc}
	1-x, &   \text{if}  & x \leq 1 \\
	-1+x, &  \text{if}  & x > 1.
	\end{array}
	\right.
	\label{absval}
	\end{equation}
	Therefore the moment generating function will be define by %the following integral
	\begin{eqnarray*}
		g(t) &=& \int_{0}^{2} e^{tx} |1 - x|\, dx \\
		&=& \int_{0}^{1} e^{tx}(1-x) \, dx + \int_{1}^{2} e^{tx} (-1+x) \, dx \\
		&=& \int_{0}^{1} e^{tx} \, dx - \int_{0}^{1} xe^{tx}\, dx -  \int_{1}^{2} e^{tx} \, dx + \int_{1}^{2} xe^{tx}\, dx \\
		&=& \left. \left[\frac{1}{t} e^{tx}\right] \right|_{0}^{1} - \left. \left[ \frac{x}{t} \cdot e^{tx} - \frac{1}{t^2} \cdot e^{tx}\right] \right|_{0}^{1} - \left. \left[\frac{1}{t} e^{tx}\right] \right|_{1}^{2} + \left. \left[ \frac{x}{t} \cdot e^{tx} - \frac{1}{t^2} \cdot e^{tx}\right] \right|_{1}^{2} \\
		&=& \frac{1}{t} e^{2t} - \frac{1}{t^2}e^{2t} + \frac{2}{t^2} e^t - \frac{1}{t^2} - \frac{1}{t}. 
	\end{eqnarray*}
	
	\noindent e) $f_X (x) = \frac{3}{8}x^2$.  
	\begin{eqnarray*}
		g(t) &=& \int_{0}^{2} e^{tx} \left(\frac{3}{8}x^2 \right)\, dx \\
		&=& \frac{3}{8} \int_{0}^{2} x^2e^{tx} \, dx.
	\end{eqnarray*}

	Integrating by parts twice, the following result is obtained
	\begin{eqnarray*}
		g(t) &=& \frac{3}{8} \int_{0}^{2} x^2e^{tx} \, dx\\
		&=& \frac{3}{8} \left. \left[e^{tx} \left( \frac{x^2}{t} - \frac{2x}{t^2} + \frac{2}{t^3}\right)\right] \right|_0^2 \\
		&=& \frac{3}{8} \left[e^{2x} \left( \frac{4t^2-4t+2}{t^3} + \frac{2}{t^3}\right)\right].
	\end{eqnarray*}

	{\bf Exercise 6, page 402}
	
	{\em Let $X$ be a continuous random variable whose characteristic function $k_X (\tau)$ is $k_X(\tau) = e^{-|\tau|}$, $-\infty < \tau < \infty$. Show directly that the density $f_X$ of $X$ is
	\begin{equation*}
		f_X(x) = \frac{1}{\pi (1+x^2)}.
	\end{equation*}
	}

	Having the characteristic function $k_X$, it is possible to determine the density function $f_X$ by Equation (\ref{density_function})
	\begin{equation}
	f_X(x) = \frac{1}{2\pi} \int_{-\infty}^{\infty} e^{-i \tau x} k_X(\tau) \, d\tau.
	\label{density_function}
	\end{equation}
	
	The characteristic function of the problem is define by an absolute value, therefore
	\begin{equation}
	k_X(\tau)= \left\{ \begin{array}{lcc}
	-\tau, &   \text{if}  & \tau \geq 0 \\
	\tau, &  \text{if}  & \tau < 0.
	\end{array}
	\right.
	\label{char_func}
	\end{equation}
	
	Using this result, the density function $f_X$ will be
	\begin{eqnarray*}
	f_{X}{(x)} &=& \frac{1}{2\pi} \int_{-\infty}^{\infty} e^{-ix\tau} e^{-|\tau|}  \, d\tau \\
	&=& \frac{1}{2\pi} \left( \int_{-\infty}^{0}e^{-ix\tau} e^{\tau}\, d\tau \right) + \frac{1}{2\pi} \left( \int_{0}^{\infty}e^{-ix\tau} e^{-\tau}\, d\tau \right) \\ 
	&=& \frac{1}{2\pi} \left( \int_{-\infty}^{0}e^{\tau(1-ix)} \, d\tau \right) + \frac{1}{2\pi} \left( \int_{0}^{\infty}e^{-\tau(1+ix)} \, d\tau \right) \\ 
	&=& \frac{1}{2\pi} \lim_{R \rightarrow \infty} \left( \int_{-R}^{0} e^{\tau(1-ix)} \, d\tau \right) +  \frac{1}{2\pi} \lim_{R \rightarrow \infty} \left( \int_{0}^{R}   e^{-\tau(1+ix)} \, d\tau \right) \\ 
	&=& \frac{1}{2\pi} \lim_{R \rightarrow \infty} \left.\left[ \left( \frac{1}{1-ix}\right) e^{\tau(1-ix)} \right] \right|_{-R}^0 +  \frac{1}{2\pi} \lim_{R \rightarrow \infty} \left.\left[ \left( -\frac{1}{1+ix}\right) e^{-\tau(1+ix)} \right] \right|_{0}^R \\ 
	&=& \frac{1}{2\pi} \left( \frac{1}{1-ix} + \frac{1}{ix+1} \right) \\ 
	&=& \frac{1}{2\pi} \cdot \frac{1+ix+1-ix}{(1-ix)(1+ix)} \\ 
	&=& \frac{1}{2\pi} \cdot \frac{2}{1-i^2x^2}  \\ 
	&=& \frac{1}{\pi(1+x^2)}.
	\end{eqnarray*}
	
	
	{\bf Exercise 10, page 403}
	
	{\em Let $X_1, X_2, \ldots, X_n$ be an independent trials process with density
	\begin{equation*}
		f(x) = \frac{1}{2}e^{-|x|}, \quad -\infty < x < \infty.
	\end{equation*} \setlength{\parskip}{0cm}
	\begin{enumerate}[a)]
		\item Find the mean and variance of $f(x)$.
		\item Find the moment generating function for $X_1, S_n, A_n$, and $S_n^*$.
		\item What can you say about the moment generating function of $S_n^*$ as $n \rightarrow \infty$?
		\item What can you say about the moment generating function of $A_n$ as $n \rightarrow \infty$?
	\end{enumerate}
	}
	
	
\bibliographystyle{plain}
\bibliography{biblio}

\end{document}
